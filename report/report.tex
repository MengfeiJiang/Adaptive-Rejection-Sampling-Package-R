\documentclass{article}\usepackage[]{graphicx}\usepackage[]{color}
%% maxwidth is the original width if it is less than linewidth
%% otherwise use linewidth (to make sure the graphics do not exceed the margin)
\makeatletter
\def\maxwidth{ %
  \ifdim\Gin@nat@width>\linewidth
    \linewidth
  \else
    \Gin@nat@width
  \fi
}
\makeatother

\definecolor{fgcolor}{rgb}{0.345, 0.345, 0.345}
\newcommand{\hlnum}[1]{\textcolor[rgb]{0.686,0.059,0.569}{#1}}%
\newcommand{\hlstr}[1]{\textcolor[rgb]{0.192,0.494,0.8}{#1}}%
\newcommand{\hlcom}[1]{\textcolor[rgb]{0.678,0.584,0.686}{\textit{#1}}}%
\newcommand{\hlopt}[1]{\textcolor[rgb]{0,0,0}{#1}}%
\newcommand{\hlstd}[1]{\textcolor[rgb]{0.345,0.345,0.345}{#1}}%
\newcommand{\hlkwa}[1]{\textcolor[rgb]{0.161,0.373,0.58}{\textbf{#1}}}%
\newcommand{\hlkwb}[1]{\textcolor[rgb]{0.69,0.353,0.396}{#1}}%
\newcommand{\hlkwc}[1]{\textcolor[rgb]{0.333,0.667,0.333}{#1}}%
\newcommand{\hlkwd}[1]{\textcolor[rgb]{0.737,0.353,0.396}{\textbf{#1}}}%

\usepackage{framed}
\makeatletter
\newenvironment{kframe}{%
 \def\at@end@of@kframe{}%
 \ifinner\ifhmode%
  \def\at@end@of@kframe{\end{minipage}}%
  \begin{minipage}{\columnwidth}%
 \fi\fi%
 \def\FrameCommand##1{\hskip\@totalleftmargin \hskip-\fboxsep
 \colorbox{shadecolor}{##1}\hskip-\fboxsep
     % There is no \\@totalrightmargin, so:
     \hskip-\linewidth \hskip-\@totalleftmargin \hskip\columnwidth}%
 \MakeFramed {\advance\hsize-\width
   \@totalleftmargin\z@ \linewidth\hsize
   \@setminipage}}%
 {\par\unskip\endMakeFramed%
 \at@end@of@kframe}
\makeatother

\definecolor{shadecolor}{rgb}{.97, .97, .97}
\definecolor{messagecolor}{rgb}{0, 0, 0}
\definecolor{warningcolor}{rgb}{1, 0, 1}
\definecolor{errorcolor}{rgb}{1, 0, 0}
\newenvironment{knitrout}{}{} % an empty environment to be redefined in TeX

\usepackage{alltt}
\usepackage{natbib}
\usepackage[unicode=true]{hyperref}
\usepackage{color}
\usepackage{amsmath, amssymb}
\usepackage{bm}
\definecolor{note_fontcolor}{rgb}{0.800781, 0.800781, 0.800781}
\usepackage{geometry}
\geometry{tmargin=1in,bmargin=1in,lmargin=1in,rmargin=1in}



\IfFileExists{upquote.sty}{\usepackage{upquote}}{}
\begin{document} 

\title{STAT 243 Final Project: Adaptive Rejection Sampling Package Development}
\author{Mengfei Jiang (SID: 26941889), Ji Wang, Zhuangdi Li, Alexander Fredh-Ojala}
\date{December 16, 2015}

\maketitle

\tableofcontents

\newpage

\section{Description}
The main function \textit{ars} takes 4 inputs, namely (1) target log-concave density 
function, (2) sample size, (3) lower bound and (4) upper bound, and returns a sample by Adaptive 
Rejection Sampling. The lower bound and the upper bound define the support for the target
density, and it is assumed that the user input the \textbf{analytically correct 
bounds} when defining the density. $-/+ \infty$ are default values respectively.
The function will then draw samples from the target density after validity checks.\\
\\
The pakcage is located in Mengfei Jiang's Github repository: \url{https://github.com/MengfeiJiang}
named as \textbf{stat243-project}.

\section{Approach}
The general approach is described by the below pseudo-code:\\
\\
Let A be a set of points $x_i, h(x_i), h'(x_i)$. The tangents at these points
define the upper hull $h_u$, the chords define the lower hull $h_l$.\\
\textit{Repeat}\\
Sample $x$ from $s(x)$, the normalized exponential of the upper hull\\
Sample $u$ from a $Unif(0,1)$\\
\indent \textit{if} $u < exp(h_l(x) - h_u(x))$ then accept $x$ (squeezing)\\
\indent \textit{else} perform the rejection test\\
\indent \indent evaluate function value $h$ at x\\
\indent \indent \textit{if} $u < exp(exp(h(x) - h_u(x)))$ \textit{then} accept $x$\\
\indent \indent \textit{else} reject $x$\\
\indent \indent \textit{endif}\\
\indent \indent update upper and lower hull by adding {$x, h(x), h'(x)$} to A\\
\indent \textit{endif}\\
\textit{until} the user required number of points are sampled\\
\\
Descriptions of each function are detailed as follows.
\subsection{Input Validity Check}
\paragraph{Check support boundaries}
Three checks are performed: lower bound is smaller than upper bound, 
density is neither $-/+ \infty$ or smaller than 0 at bounds, and density is not
0 everywhere within the bounds.
\paragraph{Check density convergence}
The function stops when the integral of the unnormalized density diverges within 
the bounds.
\paragraph{Check log-concavity of the density}
Log-concavity check is performed locally at the neighbourhood of each of the 
abscissae, and we require $h_u(x_i)$ be greather than or equal to $h_l(x_i)$. 
The function stops whenever $h_l(x_i)$ exceeds $h_u(x_i)$. Hence, log-concavity
check is performed as sampling proceeds.

\subsection{Distribution Shape Determination}
\paragraph{Mode finding}
Mode of the target density is to be used to determine the density shape and as
one of the initialization points, and hence, R's \textit{optim} function is called
to find such mode.
\paragraph{Shape determination}
We categorize the shape of a distributi on into 4 categories, (1) uniform distribution
with constant density within bounds, (2) monotonely decreasing density, (3) monotonely
increasing density, and (4) density mode occurring within bounds. R's built-in
\textit{runif} function is called to generate samples in the first shape category, while
the following steps of ARS are performed to generate samples for other categories.

\subsection{Boundary Shrink}
To avoid numerical issues, valid boundaries are shrinked to such an extent that 
the target density is not numerically zero at and within bounds if it is in the 
first place.

\subsection{Initialization}
Initial abscissae are the lower bound, the upper bound, the mode, one point to
the left of the mode and another point to the right of the mode. $h$ value, i.e.
the $log(density)$ value, and $h'$ value is calculated for each of the initial
point.

\subsection{Upper hull and lower hull}
Vectorized auxiliary functions $u$ and $l$ are such that take a vector of $x$'s 
and output the corresponding values at the upper hull and the lower hull. Each time
$h(x)$ is evaluated, the two functions are also updated. Log-concavity check is
performed for all abscissae every time the upper and lower hulls are updated.

\subsection{Intersections of upper hulls}
Intersections of the upper hulls are determined by the following formula:\\
\begin{align*}
z_i &= x_i + \frac{h(x_i) - h(x_{i+1}) + h'(x_{i+1})(x_{i+1} - x_i)}{h'(x_{i+1}) - h'(x_i)}\\
h_u(z_i) &= h'(x_i)(z_i - x_i) + h(x_i)
\end{align*}
, for $i = 1, 2, \dots, n$, and $z_0 = $ lower bound.\\
$z$'s are updated each time the upper hull and the lower hull are updated.

\subsection{Sampling}
\paragraph{CDF}
CDF up to each intersection point is calculated as follows:\\
\begin{align*}
cdf(z_0) &= 0\\
cdf(z_i) &= \frac{1}{constant}\sum_{j = 0}^{i} \frac{1}{h'(x_{j+1})}{exp(h_u(z_{j+1})) - exp(h_u(z_{j}))}
\end{align*}
, where $$constant = \sum_{j = 0}^{n-1} \frac{1}{h'(x_{j+1})}{exp(h_u(z_{j+1})) - exp(h_u(z_{j}))}$$
\paragraph{Inverse-CDF}
To sample from the customized distribution, we need to inverse the cdf. First, 
draw a vector of $Unif(0,1)$ random variables \textbf{\textit{u}}. Then, for each 
$u_j$, we find the largest $z_i$ such that cdf($z_i$) is smaller than $u_j$. Then,
the $x_j$ sampled from $u_j$ is generated as such to form candidate sample
points \textbf{\textit{x}}:\\
\[
x_j = z_i + \frac{1}{h'(x_{i+1})}log \Big[1 + \frac{h'(x_{i+1})*constant*(u_j - cdf(z_i))}{exp(h_u(z_i))} \Big]
\]
, for $i = 0, 1, 2, \dots, n-1$.
\paragraph{Squeeze test and rejection test}
A vector of $Unif(0,1)$ random variables \textbf{\textit{w}} is generated to
perform squeeze test. All the candidate points up to the first rejected point
are accepted and stored in the final result to be returned. Then, rejection test
is performed on the rejected point, which will be added to the final result if 
accepted. The upper hulls and lower hulls are then updated, and log-concavity
check is performed again.\\
If the number of accepted points in the result is smaller than the sample size,
go back to the sampling step and sample until enough.

\section{Testing}
Only main function test is detailed in the report. Unit tests are performed while
the package is being developed. The main function is tested on both log-concave and
non-log-concave distributions.
\subsection{Log-concave Distributions}
\subsubsection{Normal}
\subsubsection{Truncated Normal}
\subsubsection{Uniform}
\subsubsection{Exponential}
\subsubsection{Gamma}
\subsubsection{Beta}
\subsubsection{Logistic Distribution}
\subsubsection{Extreme Value Distribution}
\subsubsection{Laplace}
\subsubsection{Chi-squared with dof $>=$ 2}
\subsubsection{Weibull Distribution}


\subsection{Non Log-concave Distributions}
\subsubsection{Chi-squared with dof = 1}
\subsubsection{Student's t Distribution}
\subsubsection{Cauchy distribution}
\subsubsection{Pareto Distribution}
\subsubsection{F-Distribution}


\section{Logistics}
\subsection{Code}
\subsection{Testing}
\subsection{Report}
\end{document}
